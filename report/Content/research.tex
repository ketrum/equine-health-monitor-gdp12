\chapter{Research}


\section{Grass Sickness}
A literature research and discussion sessions with experts were done to acquire information about which sensor data could be useful and how it could be used to diagnose grass sickness.  


\section{Literature Search}
Grass sickness (equine dysautonomia) is defined as a disease of equidae characterised by damage to autonomic, enteric and somatic neurons which cause low gastrointestinal motility and paralysis of the gut as a result of this [citations]. It can be classified as acute, subacute and chronic grass sickness based on its severity. Acute grass sickness has a sudden onset while chronic grass sickness shows clinical signs gradually. Symptoms of acute grass sickness include severe colic, gastrointestinal stasis and increased heart rate (70-120 beats per minute). Horses with subacute grass sickness exhibit less severe gastrointestinal stasis and increased heart rate (60-80 beats per minute). Chronic grass sickness cause gradual weight loss, increased heart rate (50-60 beats per minute) and milder gastrointestinal stasis. Horses with acute grass sickness die within 48 hours after the onset of clinical signs while the ones with chronic grass sickness can be saved with intensive care.
\TODO{Citations}

Since low gastrointestinal motility is an indicator of grass sickness, borborygmi can be used for diagnosis. Borborygmi (gut sounds) is the noise caused by gas and fluid pushed in the gastrointestinal system. It shows the status of the gastrointestinal system [citations]. The frequency of borborygmi is 2 to 4 times a minute in normal conditions. Although borborygmi may be an indication of the problems with the gastrointestinal tract, it cannot not provide enough evidence to detect them.
\TODO{Citations}

\section{Discussions with Bioscientists}
The team had meetings with Dr John Chad and Dr Neil Smyth TODO from which dept? uni? about the project. A presentation about horse gastrointestinal system and gut sound monitoring was given by Dr Smyth. The outcomes of the meetings will be explained in this section. \TODO{Cleanup}

For a successful abdominal examination, four different sites should be auscultated: left and right lower and upper parts of the abdomen [citation]. However, the most loud sound can be heard at the lower right part of the body where the large intestine is. 

The proposed system contains mobile monitoring devices attached to the horses. It is not convenient to use a mobile device with four audio recording units to auscultate four different sites of the body since it would require additional wires for communication. Therefore the monitoring device which contains one single audio recording unit can be placed at the lower right part of the abdomen.

The situations which may cause too loud or too quiet gut sound vary [citation]. The complexity of diagnosing grass sickness makes an automated solution unfeasible in the scope of this project. Therefore, the goal of the project was shifted from diagnosis to profiling of vital signals.   
\TODO{Citations}