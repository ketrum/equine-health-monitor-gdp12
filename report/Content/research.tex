\chapter{Research}
\label{chap:research}

\section{Grass Sickness}
Since an important goal for the project was to determine the feasibility of automatically diagnosing grass sickness, literature research and a discussion session with experts were conducted to acquire more theoretical and practical information about grass sickness disease. This information was very important to decide which sensor data could be useful and how it could be used to provide indicators for a diagnosis of grass sickness disease.  


\section{Literature Search}
Information about normal vital signs of an adult horse and the symptoms of grass sickness were collected through literature research. This gave us an idea about how the collected data could be used either for automation or by the medical experts to make a diagnosis. Here we attempt to summarize the knowledge we obtained on the definition, symptoms and potential results of the disease, as well as the relevant vital signs from healthy horses for comparison.

Grass sickness (equine dysautonomia) is defined as a disease of equids characterised by damage to autonomic, enteric and somatic neurons which cause low gastrointestinal motility and paralysis of the gut as a result of this. It can be classified as acute, subacute and chronic grass sickness based on its severity. Acute grass sickness has a sudden onset while chronic grass sickness shows clinical signs gradually. Horses with acute grass sickness die within 48 hours after the onset of clinical signs while the ones with chronic grass sickness can be saved with intensive care. \cite{robinson2009current}, \cite{edwards2010edaphic}.

Symptoms of grass sickness include colic, gastrointestinal stasis and increased heart rate. Therefore heart rate data could be useful to detect grass sickness. Heart rate of a horse with acute grass sickness is typically 70-120 bpm (beats per minute), whereas horses with subacute grass sickness have a heart rate of 60-80 bpm and the ones with chronic grass sickness have that of 50-60 bpm. Heart rate of an adult horse changes between 30-40 bpm under normal conditions. \cite{corley2009equine}, \cite{robinson2009current}

Since low gastrointestinal motility is also an indicator of grass sickness, borborygmi (gut sounds) can be used for diagnosis. Borborygmi is the noise caused by gas and fluid pushed in the gastrointestinal system. It shows the status of the gastrointestinal system. The frequency of borborygmi is 2 to 4 times a minute in normal conditions. Although borborygmi may be an indication of the problems with the gastrointestinal tract, it cannot not provide itself enough evidence to detect them. \cite{corley2009equine}


\section{Discussions with Bioscientists}
\label{sec:research_discussion}
The team had meetings with Dr. John Chad and Dr. Neil Smyth about the project. A presentation about horse gastrointestinal system and gut sound monitoring was given by Dr. Smyth. The outcomes of the meetings will be explained in this section. 

Dr. Smyth explained that for performing a successful abdominal examination, four different sites should be auscultated: left and right lower and upper parts of the abdomen. However, the loudest can be heard at the lower right part of the abdomen where the large intestine is. 

The proposed system contains a single mobile monitoring device attached to each horse. It is not convenient to use a mobile device with four audio recording units to auscultate four different sites of the body since it would require additional wires for communication. Therefore the monitoring device which contains a single audio recording unit can be placed at the lower right part of the abdomen.

Dr. Smyth stated that the situations which may cause too loud or too quiet gut sound can vary. There is not a enough data available on the web regarding the properties or signal characteristics of gut sounds and the relation of those with common horse illnesses and these tasks are normally performed by specialists based on empirical knowledge. The complexity of diagnosing grass sickness makes an automated solution unfeasible in the scope of this project. 

Therefore, the main goal of the project was shifted from diagnosis to health data collection and profiling of vital signals. The scalability and wireless communication features of the Equine Health Monitor make it suitable for building biological profiles by collecting the vital signals of groups of horses. This constitutes an innovative and practical tool for veterinary and biological research as well as clinical practice. 
